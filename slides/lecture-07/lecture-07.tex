\documentclass{beamer}\usepackage[]{graphicx}\usepackage[]{color}
% maxwidth is the original width if it is less than linewidth
% otherwise use linewidth (to make sure the graphics do not exceed the margin)
\makeatletter
\def\maxwidth{ %
  \ifdim\Gin@nat@width>\linewidth
    \linewidth
  \else
    \Gin@nat@width
  \fi
}
\makeatother

\definecolor{fgcolor}{rgb}{0.345, 0.345, 0.345}
\newcommand{\hlnum}[1]{\textcolor[rgb]{0.686,0.059,0.569}{#1}}%
\newcommand{\hlstr}[1]{\textcolor[rgb]{0.192,0.494,0.8}{#1}}%
\newcommand{\hlcom}[1]{\textcolor[rgb]{0.678,0.584,0.686}{\textit{#1}}}%
\newcommand{\hlopt}[1]{\textcolor[rgb]{0,0,0}{#1}}%
\newcommand{\hlstd}[1]{\textcolor[rgb]{0.345,0.345,0.345}{#1}}%
\newcommand{\hlkwa}[1]{\textcolor[rgb]{0.161,0.373,0.58}{\textbf{#1}}}%
\newcommand{\hlkwb}[1]{\textcolor[rgb]{0.69,0.353,0.396}{#1}}%
\newcommand{\hlkwc}[1]{\textcolor[rgb]{0.333,0.667,0.333}{#1}}%
\newcommand{\hlkwd}[1]{\textcolor[rgb]{0.737,0.353,0.396}{\textbf{#1}}}%
\let\hlipl\hlkwb

\usepackage{framed}
\makeatletter
\newenvironment{kframe}{%
 \def\at@end@of@kframe{}%
 \ifinner\ifhmode%
  \def\at@end@of@kframe{\end{minipage}}%
  \begin{minipage}{\columnwidth}%
 \fi\fi%
 \def\FrameCommand##1{\hskip\@totalleftmargin \hskip-\fboxsep
 \colorbox{shadecolor}{##1}\hskip-\fboxsep
     % There is no \\@totalrightmargin, so:
     \hskip-\linewidth \hskip-\@totalleftmargin \hskip\columnwidth}%
 \MakeFramed {\advance\hsize-\width
   \@totalleftmargin\z@ \linewidth\hsize
   \@setminipage}}%
 {\par\unskip\endMakeFramed%
 \at@end@of@kframe}
\makeatother

\definecolor{shadecolor}{rgb}{.97, .97, .97}
\definecolor{messagecolor}{rgb}{0, 0, 0}
\definecolor{warningcolor}{rgb}{1, 0, 1}
\definecolor{errorcolor}{rgb}{1, 0, 0}
\newenvironment{knitrout}{}{} % an empty environment to be redefined in TeX

\usepackage{alltt}
\usepackage{../371g-slides}
\title{Decision Trees 2}
\subtitle{Lecture 7}
\author{STA 371G}
\IfFileExists{upquote.sty}{\usepackage{upquote}}{}
\begin{document}
  
  

  \frame{\maketitle}

  % Show outline at beginning of each section
  \AtBeginSection[]{
    \begin{frame}<beamer>
      \tableofcontents[currentsection]
    \end{frame}
  }

  %%%%%%% Slides start here %%%%%%%

  \begin{darkframes}
    

    \begin{frame}{Oil drilling}
      \begin{itemize}[<+->]
        \item Suppose you are planning to drill for oil in a newly-discovered field.
        \item Setting up the drilling equipment costs \$1M.
        \item If you strike oil, you will generate \$10M.
        \item But there's only a 15\% chance that a random oil field has oil.
      \end{itemize}
      
    \end{frame}

    \begin{frame}
      \begin{center}
        \begin{tikzpicture}
        [
          grow                    = right,
          sibling distance        = 6em,
          level distance          = 10em,
          edge from parent/.style = {draw, -latex},
          every node/.style       = {font=\footnotesize},
          sloped
        ]
        \node [decision] {}
          child {
            node [leaf] {$0$}
            edge from parent node [below] {Don't drill}
          }
          child {
            node [prob] {}
            child {
              node [leaf] { $-1$ }
              edge from parent node [above] {No oil}
              edge from parent node [below] {$p=0.85$}
            }
            child {
              node [leaf] { $9$ }
              edge from parent node [above] {Strike oil}
              edge from parent node [below] {$p=0.15$}
            }
            edge from parent node [above] {Drill}
          };
        \end{tikzpicture}
      \end{center}
    \end{frame}

    \begin{frame}
      
      \begin{center}
        \begin{tikzpicture}
        [
          grow                    = right,
          sibling distance        = 6em,
          level distance          = 10em,
          edge from parent/.style = {draw, -latex},
          every node/.style       = {font=\footnotesize},
          sloped
        ]
        \node [decision] {$0.5$}
          child {
            node [leaf] {$0$}
            edge from parent node [below] {Don't drill}
          }
          child {
            node [prob] {$0.5$}
            child {
              node [leaf] { $-1$ }
              edge from parent node [above] {No oil}
              edge from parent node [below] {$p=0.85$}
            }
            child {
              node [leaf] { $9$ }
              edge from parent node [above] {Strike oil}
              edge from parent node [below] {$p=0.15$}
            }
            edge from parent node [above] {Drill}
          };
        \end{tikzpicture}
        \pause\bigskip

        Without any information, we expect to make \$0.5M by deciding to drill.
      \end{center}
    \end{frame}

    \begin{frame}{How much is perfect information worth?}
      \begin{itemize}[<+->]
        \item Suppose we have a psychic that gives us \emph{perfect information} about whether the field has oil or not.
        \item The psychic is always right!
      \end{itemize}
    \end{frame}

    \begin{frame}
      

      \begin{center}
        \begin{tikzpicture}
        [
          grow                    = right,
          sibling distance        = 11em,
          level distance          = 7em,
          edge from parent/.style = {draw, -latex},
          every node/.style       = {font=\footnotesize},
          sloped
        ]
        \node [prob] {$1.35$}
          child {
            node [decision] {$0$}
            child [sibling distance=7em] {
              node [prob] {$-1$}
              child [sibling distance=7em]  {
                node [leaf] {$9$}
                edge from parent node [above] {Strike oil}
                edge from parent node [below] {$p = 0$}
              }
              child {
                node [leaf] {$-1$}
                edge from parent node [above] {No oil}
                edge from parent node [below] {$p = 1$}
              }
              edge from parent node [above] {Drill}
            }
            child [sibling distance=7em] {
              node [leaf] {0}
              edge from parent node [above] {Don't drill}
            }
            edge from parent node [above] {Psychic says dry}
            edge from parent node [below] {$p = 0.85$}
          }
          child {
            node [decision] {$9$}
            child [sibling distance=7em] {
              node [prob] {$9$}
              child {
                node [leaf] {$9$}
                edge from parent node [above] {Strike oil}
                edge from parent node [below] {$p = 1$}
              }
              child {
                node [leaf] {$-1$}
                edge from parent node [above] {No oil}
                edge from parent node [below] {$p = 0$}
              }
              edge from parent node [above] {Drill}
            }
            child [sibling distance=7em] {
              node [leaf] {0}
              edge from parent node [above] {Don't drill}
            }
            edge from parent node [above] {Psychic says oil}
            edge from parent node [below] {$p = 0.15$}
          };
        \end{tikzpicture}
      \end{center}
      \note{Before showing this tree, build up the skeleton of the}
      \note{tree on the board, and use LC Q3 to get students}
      \note{to fill it out. Then use LC Q4 either before or after.}
    \end{frame}

    \begin{frame}
      
      \begin{itemize}[<+->]
        \item Suppose we commission a geological survey of the field; this is \emph{imperfect information}.
        \item When a field is oil-rich, the survey will indicate this 60\% of the time.
        \item When a field is dry, the survey will indicate this 80\% of the time.
        \item How much should we be willing to pay for the survey?
      \end{itemize}
      \note{LC Q5-Q6 to ensure students understand the numbers.}
    \end{frame}

    \begin{frame}
      \begin{center}
        \begin{tikzpicture}
        [
          grow                    = right,
          sibling distance        = 11em,
          level distance          = 7em,
          edge from parent/.style = {draw, -latex},
          every node/.style       = {font=\footnotesize},
          sloped
        ]
        \node [prob] {}
          child {
            node [decision] {}
            child [sibling distance=7em] {
              node [prob] {}
              child [sibling distance=7em]  {
                node [leaf] {$9$}
                edge from parent node [above] {Strike oil}
              }
              child {
                node [leaf] {$-1$}
                edge from parent node [above] {No oil}
              }
              edge from parent node [above] {Drill}
            }
            child [sibling distance=7em]  {
              node [leaf] {0}
              edge from parent node [above] {Don't drill}
            }
            edge from parent node [above] {Survey says dry}
          }
          child {
            node [decision] {}
            child [sibling distance=7em] {
              node [prob] {}
              child {
                node [leaf] {$9$}
                edge from parent node [above] {Strike oil}
              }
              child {
                node [leaf] {$-1$}
                edge from parent node [above] {No oil}
                edge from parent node [below] {$p=p_1$}
              }
              edge from parent node [above] {Drill}
            }
            child [sibling distance=7em] {
              node [leaf] {0}
              edge from parent node [above] {Don't drill}
            }
            edge from parent node [above] {Survey says oil}
            edge from parent node [below] {$p=p_2$}
          };
        \end{tikzpicture}
      \end{center}
      \note{LC Q7-Q8 to identify p_1 and p_2}
    \end{frame}

    \begin{frame}
      Let's say $O$ is the event that you would strike oil in the oilfield, and $S$ is the event that the survey indicates there is oil present.

      \bigskip\pause

      There's only a 15\% chance that a random oil field has oil.

      \[  P(O) = 0.15, \qquad P(O^c) = 0.85 \]

      \bigskip\pause

      When a field is oil-rich, the survey will indicate this 60\% of the time.

      \pause

      \[ P(S|O) = 0.6, \qquad\pause P(S^c|O) = 0.4 \]

      \pause

      When a field is dry, the survey will indicate this 80\% of the time.

      \pause

      \[ P(S|O^c) = 0.2, \qquad\pause P(S^c|O^c) = 0.8 \]
    \end{frame}

    \begin{frame}
      \begin{center}
        \begin{tikzpicture}
        [
          grow                    = right,
          sibling distance        = 11em,
          level distance          = 7em,
          edge from parent/.style = {draw, -latex},
          every node/.style       = {font=\footnotesize},
          sloped
        ]
        \node [prob] {}
          child {
            node [decision] {}
            child [sibling distance=7em] {
              node [prob] {}
              child [sibling distance=7em]  {
                node [leaf] {$9$}
                edge from parent node [above] {Strike oil}
                edge from parent node [below] {$P(O|S^c)$}
              }
              child {
                node [leaf] {$-1$}
                edge from parent node [above] {No oil}
                edge from parent node [below] {$P(O^c|S^c)$}
              }
              edge from parent node [above] {Drill}
            }
            child [sibling distance=7em] {
              node [leaf] {0}
              edge from parent node [above] {Don't drill}
            }
            edge from parent node [above] {Survey says dry}
            edge from parent node [below] {$P(S^c)$}
          }
          child {
            node [decision] {}
            child [sibling distance=7em] {
              node [prob] {}
              child {
                node [leaf] {$9$}
                edge from parent node [above] {Strike oil}
                edge from parent node [below] {$P(O|S)$}
              }
              child {
                node [leaf] {$-1$}
                edge from parent node [above] {No oil}
                edge from parent node [below] {$P(O^c|S)$}
              }
              edge from parent node [above] {Drill}
            }
            child [sibling distance=7em] {
              node [leaf] {0}
              edge from parent node [above] {Don't drill}
            }
            edge from parent node [above] {Survey says oil}
            edge from parent node [below] {$P(S)$}
          };
        \end{tikzpicture}
      \end{center}
    \end{frame}

    \begin{frame}
      Let's calculate the required probabilities using Bayes' rule:
      \begin{align*}
        P(O|S) &= \frac{P(S|O)P(O)}{P(S|O)P(O) + P(S|O^c)P(O^c)} \\
        &= \frac{(0.6)(0.15)}{(0.6)(0.15) + (0.2)(0.85)} \\
        &= 0.35
      \end{align*}
      \begin{align*}
        P(O^c|S) &= \frac{P(S|O^c)P(O^c)}{P(S|O^c)P(O^c) + P(S|O)P(O)} \\
        &= \frac{(0.2)(0.85)}{(0.2)(0.85) + (0.6)(0.15)} \\
        &= 0.65
      \end{align*}
      \note{LC Q9 to compute one of the other probabilities}
    \end{frame}

    \begin{frame}
      We also need to know the probability that the survey will indicate oil:
      \begin{align*}
        P(S) &= P(\text{$S$ and $O$}) + P(\text{$S$ and $O^c$}) \\
        &= P(S|O)P(O) + P(S|O^c)P(O^c) \\
        &= (0.6)(0.15) + (0.2)(0.85) \\
        &= 0.26
      \end{align*}
    \end{frame}

    \begin{frame}
      \begin{center}
        \begin{tikzpicture}
        [
          grow                    = right,
          sibling distance        = 11em,
          level distance          = 7em,
          edge from parent/.style = {draw, -latex},
          every node/.style       = {font=\footnotesize},
          sloped
        ]
        \node [prob] {$0.64$}
          child {
            node [decision] {$0$}
            child [sibling distance=7em] {
              node [prob] {$-0.19$}
              child [sibling distance=7em]  {
                node [leaf] {$9$}
                edge from parent node [above] {Strike oil}
                edge from parent node [below] {$p = 0.08$}
              }
              child {
                node [leaf] {$-1$}
                edge from parent node [above] {No oil}
                edge from parent node [below] {$p = 0.92$}
              }
              edge from parent node [above] {Drill}
            }
            child [sibling distance=7em] {
              node [leaf] {0}
              edge from parent node [above] {Don't drill}
            }
            edge from parent node [above] {Survey says dry}
            edge from parent node [below] {$p = 0.74$}
          }
          child {
            node [decision] {$2.46$}
            child [sibling distance=7em] {
              node [prob] {$2.46$}
              child {
                node [leaf] {$9$}
                edge from parent node [above] {Strike oil}
                edge from parent node [below] {$p = 0.35$}
              }
              child {
                node [leaf] {$-1$}
                edge from parent node [above] {No oil}
                edge from parent node [below] {$p = 0.65$}
              }
              edge from parent node [above] {Drill}
            }
            child [sibling distance=7em] {
              node [leaf] {0}
              edge from parent node [above] {Don't drill}
            }
            edge from parent node [above] {Survey says oil}
            edge from parent node [below] {$p = 0.26$}
          };
        \end{tikzpicture}
      \end{center}
      \note{LC Q10 to compute EVSI, Q11 to explore factors impacting EVSI}
    \end{frame}

    \begin{frame}
      

      \begin{itemize}[<+->]
        \item The information from the survey is worth the difference between these two trees ($0.64 - 0.5 = \$0.14$M); this is EVSI (Expected Value of Sample Information)
        \item Suppose we had a better survey, which correctly identifies oil-rich fields 95\% of the time, and correctly identifies dry fields 90\% of the time.
      \end{itemize}
      \note{LC Q12 if you have time to have them calculate the new tree.}
    \end{frame}

    \begin{frame}
      \begin{center}
        \begin{tikzpicture}
        [
          grow                    = right,
          sibling distance        = 11em,
          level distance          = 7em,
          edge from parent/.style = {draw, -latex},
          every node/.style       = {font=\footnotesize},
          sloped
        ]
        \node [prob] {$1.2$}
          child {
            node [decision] {$0$}
            child [sibling distance=7em] {
              node [prob] {$-0.9$}
              child [sibling distance=7em]  {
                node [leaf] {$9$}
                edge from parent node [above] {Strike oil}
                edge from parent node [below] {$p = 0.01$}
              }
              child {
                node [leaf] {$-1$}
                edge from parent node [above] {No oil}
                edge from parent node [below] {$p = 0.99$}
              }
              edge from parent node [above] {Drill}
            }
            child [sibling distance=7em] {
              node [leaf] {0}
              edge from parent node [above] {Don't drill}
            }
            edge from parent node [above] {Survey says dry}
            edge from parent node [below] {$p = 0.77$}
          }
          child {
            node [decision] {$5.26$}
            child [sibling distance=7em] {
              node [prob] {$5.26$}
              child {
                node [leaf] {$9$}
                edge from parent node [above] {Strike oil}
                edge from parent node [below] {$p = 0.63$}
              }
              child {
                node [leaf] {$-1$}
                edge from parent node [above] {No oil}
                edge from parent node [below] {$p = 0.37$}
              }
              edge from parent node [above] {Drill}
            }
            child [sibling distance=7em] {
              node [leaf] {0}
              edge from parent node [above] {Don't drill}
            }
            edge from parent node [above] {Survey says oil}
            edge from parent node [below] {$p = 0.23$}
          };
        \end{tikzpicture}
      \end{center}
    \end{frame}

    \begin{frame}{Summary}
      \begin{itemize}[<+->]
        \item With no information, the expected value of the tree was \$0.5M.
        \item With perfect information (the psychic), the expected value of the tree was \$1.35M. The EVPI (expected value of perfect information) is $1.35 - 0.5 = \$0.85$M.
        \begin{itemize}
          \item No survey (or any other kind of information about this oil field) could ever be worth more than \$0.85M.
        \end{itemize}
        \item With imperfect information (the better survey), the expected value of the tree was \$1.2M. The EVSI (expected value of sample information) is $1.2 - 0.5 = \$0.7$M.
        \begin{itemize}
          \item It's worth paying up to \$0.7M for this particular survey.
        \end{itemize}

      \end{itemize}
    \end{frame}
  \end{darkframes}
\end{document}
