\documentclass{beamer}\usepackage[]{graphicx}\usepackage[]{color}
% maxwidth is the original width if it is less than linewidth
% otherwise use linewidth (to make sure the graphics do not exceed the margin)
\makeatletter
\def\maxwidth{ %
  \ifdim\Gin@nat@width>\linewidth
    \linewidth
  \else
    \Gin@nat@width
  \fi
}
\makeatother

\definecolor{fgcolor}{rgb}{1, 0.894, 0.769}
\newcommand{\hlnum}[1]{\textcolor[rgb]{0.824,0.412,0.118}{#1}}%
\newcommand{\hlstr}[1]{\textcolor[rgb]{1,0.894,0.71}{#1}}%
\newcommand{\hlcom}[1]{\textcolor[rgb]{0.824,0.706,0.549}{#1}}%
\newcommand{\hlopt}[1]{\textcolor[rgb]{1,0.894,0.769}{#1}}%
\newcommand{\hlstd}[1]{\textcolor[rgb]{1,0.894,0.769}{#1}}%
\newcommand{\hlkwa}[1]{\textcolor[rgb]{0.941,0.902,0.549}{#1}}%
\newcommand{\hlkwb}[1]{\textcolor[rgb]{0.804,0.776,0.451}{#1}}%
\newcommand{\hlkwc}[1]{\textcolor[rgb]{0.78,0.941,0.545}{#1}}%
\newcommand{\hlkwd}[1]{\textcolor[rgb]{1,0.78,0.769}{#1}}%
\let\hlipl\hlkwb

\usepackage{framed}
\makeatletter
\newenvironment{kframe}{%
 \def\at@end@of@kframe{}%
 \ifinner\ifhmode%
  \def\at@end@of@kframe{\end{minipage}}%
  \begin{minipage}{\columnwidth}%
 \fi\fi%
 \def\FrameCommand##1{\hskip\@totalleftmargin \hskip-\fboxsep
 \colorbox{shadecolor}{##1}\hskip-\fboxsep
     % There is no \\@totalrightmargin, so:
     \hskip-\linewidth \hskip-\@totalleftmargin \hskip\columnwidth}%
 \MakeFramed {\advance\hsize-\width
   \@totalleftmargin\z@ \linewidth\hsize
   \@setminipage}}%
 {\par\unskip\endMakeFramed%
 \at@end@of@kframe}
\makeatother

\definecolor{shadecolor}{rgb}{.97, .97, .97}
\definecolor{messagecolor}{rgb}{0, 0, 0}
\definecolor{warningcolor}{rgb}{1, 0, 1}
\definecolor{errorcolor}{rgb}{1, 0, 0}
\newenvironment{knitrout}{}{} % an empty environment to be redefined in TeX

\usepackage{alltt}
\usepackage{../371g-slides}
\title{What can go wrong, and how to fix it 2}
\subtitle{Lecture 14}
\author{STA 371G}
\IfFileExists{upquote.sty}{\usepackage{upquote}}{}
\begin{document}

  

  \frame{\maketitle}

  % Show outline at beginning of each section
  \AtBeginSection[]{
    \begin{frame}<beamer>
      \tableofcontents[currentsection]
    \end{frame}
  }

  %%%%%%% Slides start here %%%%%%%

  \begin{darkframes}
    \begin{frame}
      \fullpagepicture{debbie-downer}
    \end{frame}

    \begin{frame}{Mileage efficiency data set}
      The data set \texttt{cars} contains specs for 392 different cars. We'll focus on two variables:
      \begin{itemize}
        \item \textbf{MPG} is fuel efficiency, measured in miles per gallon
        \item \textbf{Weight} is the weight of the car, in pounds
      \end{itemize}
    \end{frame}

    \begin{frame}[fragile]{What problems do you see here?}
\begin{knitrout}
\definecolor{shadecolor}{rgb}{0.137, 0.137, 0.137}\color{fgcolor}
\includegraphics[width=\maxwidth]{/tmp/figures/unnamed-chunk-2-1} 

\end{knitrout}
    \end{frame}

    \begin{frame}[fragile]{Using transformations to fix problems}
      \begin{itemize}
        \item Sometimes, a violation of regression assumptions can be fixed by transforming one or the other of the variables (or both).
        \item When we transform a variable, we have to also transform our interpretation of the equation.
      \end{itemize}
    \end{frame}

    \begin{frame}[fragile]{A bad example}
\begin{knitrout}
\definecolor{shadecolor}{rgb}{0.137, 0.137, 0.137}\color{fgcolor}\begin{kframe}
\begin{alltt}
\hlstd{cars}\hlopt{$}\hlstd{WeightSq} \hlkwb{<-} \hlstd{cars}\hlopt{$}\hlstd{Weight}\hlopt{^}\hlnum{2}
\hlkwd{plot}\hlstd{(MPG} \hlopt{~} \hlstd{WeightSq,} \hlkwc{data}\hlstd{=cars,} \hlkwc{pch}\hlstd{=}\hlnum{16}\hlstd{,} \hlkwc{col}\hlstd{=}\hlstr{"lightblue"}\hlstd{)}
\hlstd{sq.model} \hlkwb{<-} \hlkwd{lm}\hlstd{(MPG} \hlopt{~} \hlstd{WeightSq,} \hlkwc{data}\hlstd{=cars)}
\hlkwd{abline}\hlstd{(sq.model,} \hlkwc{col}\hlstd{=}\hlstr{"orange"}\hlstd{,} \hlkwc{lwd}\hlstd{=}\hlnum{4}\hlstd{)}
\end{alltt}
\end{kframe}
\includegraphics[width=\maxwidth]{/tmp/figures/unnamed-chunk-3-1} 

\end{knitrout}
    \end{frame}

    \begin{frame}[fragile]{The log transformation}
      The \alert{log} transformation is frequently useful in regression, because many nonlinear relationships are naturally exponential.
    \end{frame}

    \begin{frame}
      \begin{center}
        \movie{\includegraphics[width=\textwidth]{log}}{log.mp4}
      \end{center}
    \end{frame}

    \begin{frame}[fragile]{The log transformation}
      The \alert{log} transformation is frequently useful in regression, because many nonlinear relationships are naturally exponential.

      \bigskip

      \begin{itemize}
        \item $\log_b x = y$ when $b^y = x$
        \item For example, $\log_{10} 1000 = 3$, $\log_{10} 100 = 2$, and $\log_{10} 10 = 1$
        \item The natural log is $\log_e$, where $e \approx 2.72$ --- when we say ``log'' we will usually mean ``natural log'' (although for our purposes the base doesn't matter)
      \end{itemize}
    \end{frame}

    \begin{frame}[fragile]{Applying a log transformation}
      \fontsm
\begin{knitrout}
\definecolor{shadecolor}{rgb}{0.137, 0.137, 0.137}\color{fgcolor}\begin{kframe}
\begin{alltt}
\hlstd{cars}\hlopt{$}\hlstd{LogWeight} \hlkwb{<-} \hlkwd{log}\hlstd{(cars}\hlopt{$}\hlstd{Weight)}
\hlkwd{plot}\hlstd{(MPG} \hlopt{~} \hlstd{LogWeight,} \hlkwc{data}\hlstd{=cars,} \hlkwc{pch}\hlstd{=}\hlnum{16}\hlstd{,} \hlkwc{col}\hlstd{=}\hlstr{"lightblue"}\hlstd{)}
\hlstd{log.model} \hlkwb{<-} \hlkwd{lm}\hlstd{(MPG} \hlopt{~} \hlstd{LogWeight,} \hlkwc{data}\hlstd{=cars)}
\hlkwd{abline}\hlstd{(log.model,} \hlkwc{col}\hlstd{=}\hlstr{"orange"}\hlstd{,} \hlkwc{lwd}\hlstd{=}\hlnum{4}\hlstd{)}
\end{alltt}
\end{kframe}
\includegraphics[width=\maxwidth]{/tmp/figures/unnamed-chunk-4-1} 

\end{knitrout}
    \end{frame}

    \begin{frame}[fragile]{Checking assumptions of our new model}
      \fontsm
\begin{knitrout}
\definecolor{shadecolor}{rgb}{0.137, 0.137, 0.137}\color{fgcolor}\begin{kframe}
\begin{alltt}
\hlkwd{plot}\hlstd{(cars}\hlopt{$}\hlstd{LogWeight,} \hlkwd{residuals}\hlstd{(log.model),} \hlkwc{pch}\hlstd{=}\hlnum{16}\hlstd{,} \hlkwc{col}\hlstd{=}\hlstr{"pink"}\hlstd{)}
\end{alltt}
\end{kframe}
\includegraphics[width=\maxwidth]{/tmp/figures/unnamed-chunk-5-1} 

\end{knitrout}
      Linearity looks good, but homoscedasticity is still not satisfied!
    \end{frame}

    \begin{frame}[fragile]{Applying a second log transformation}
      \fontsm
\begin{knitrout}
\definecolor{shadecolor}{rgb}{0.137, 0.137, 0.137}\color{fgcolor}\begin{kframe}
\begin{alltt}
\hlstd{cars}\hlopt{$}\hlstd{LogMPG} \hlkwb{<-} \hlkwd{log}\hlstd{(cars}\hlopt{$}\hlstd{MPG)}
\hlkwd{plot}\hlstd{(LogMPG} \hlopt{~} \hlstd{LogWeight,} \hlkwc{data}\hlstd{=cars,} \hlkwc{pch}\hlstd{=}\hlnum{16}\hlstd{,} \hlkwc{col}\hlstd{=}\hlstr{"lightblue"}\hlstd{)}
\hlstd{log.log.model} \hlkwb{<-} \hlkwd{lm}\hlstd{(LogMPG} \hlopt{~} \hlstd{LogWeight,} \hlkwc{data}\hlstd{=cars)}
\hlkwd{abline}\hlstd{(log.log.model,} \hlkwc{col}\hlstd{=}\hlstr{"orange"}\hlstd{,} \hlkwc{lwd}\hlstd{=}\hlnum{4}\hlstd{)}
\end{alltt}
\end{kframe}
\includegraphics[width=\maxwidth]{/tmp/figures/unnamed-chunk-6-1} 

\end{knitrout}
    \end{frame}

    \begin{frame}[fragile]{Checking assumptions of the log-log model}
      \fontsm
\begin{knitrout}
\definecolor{shadecolor}{rgb}{0.137, 0.137, 0.137}\color{fgcolor}\begin{kframe}
\begin{alltt}
\hlkwd{plot}\hlstd{(cars}\hlopt{$}\hlstd{LogWeight,} \hlkwd{residuals}\hlstd{(log.log.model),} \hlkwc{pch}\hlstd{=}\hlnum{16}\hlstd{,} \hlkwc{col}\hlstd{=}\hlstr{"pink"}\hlstd{)}
\end{alltt}
\end{kframe}
\includegraphics[width=\maxwidth]{/tmp/figures/unnamed-chunk-7-1} 

\end{knitrout}
      Much better---transforming MPG to log(MPG) gives us both linearity and homoscedasticity!
    \end{frame}

    \begin{frame}{Another way to transform}
      \begin{itemize}
        \item Instead of transforming using logs, we can also transform using the reciprocal ($x \to 1/x$)
        \item Let's transform $Y$ by making $\text{MPG} \to 1/\text{MPG}$
      \end{itemize}
    \end{frame}

    \begin{frame}[fragile]
\begin{knitrout}
\definecolor{shadecolor}{rgb}{0.137, 0.137, 0.137}\color{fgcolor}\begin{kframe}
\begin{alltt}
\hlstd{cars}\hlopt{$}\hlstd{rMPG} \hlkwb{<-} \hlnum{1}\hlopt{/}\hlstd{(cars}\hlopt{$}\hlstd{MPG)}
\hlkwd{plot}\hlstd{(rMPG} \hlopt{~} \hlstd{Weight,} \hlkwc{data}\hlstd{=cars,} \hlkwc{pch}\hlstd{=}\hlnum{16}\hlstd{,} \hlkwc{col}\hlstd{=}\hlstr{"lightblue"}\hlstd{)}
\hlstd{recip.model} \hlkwb{<-} \hlkwd{lm}\hlstd{(rMPG} \hlopt{~} \hlstd{Weight,} \hlkwc{data}\hlstd{=cars)}
\hlkwd{abline}\hlstd{(recip.model,} \hlkwc{col}\hlstd{=}\hlstr{"orange"}\hlstd{,} \hlkwc{lwd}\hlstd{=}\hlnum{4}\hlstd{)}
\end{alltt}
\end{kframe}
\includegraphics[width=\maxwidth]{/tmp/figures/unnamed-chunk-8-1} 

\end{knitrout}
    \end{frame}

    \begin{frame}[fragile]{Checking assumptions of the reciprocal model}
      \fontsm
\begin{knitrout}
\definecolor{shadecolor}{rgb}{0.137, 0.137, 0.137}\color{fgcolor}\begin{kframe}
\begin{alltt}
\hlkwd{plot}\hlstd{(cars}\hlopt{$}\hlstd{Weight,} \hlkwd{residuals}\hlstd{(recip.model),} \hlkwc{pch}\hlstd{=}\hlnum{16}\hlstd{,} \hlkwc{col}\hlstd{=}\hlstr{"pink"}\hlstd{)}
\end{alltt}
\end{kframe}
\includegraphics[width=\maxwidth]{/tmp/figures/unnamed-chunk-9-1} 

\end{knitrout}
    \end{frame}

    \begin{frame}{Which transformation is better?}
      \begin{itemize}[<+->]
        \item We could compare $R^2$ values: $R^2 = 0.78$ for the reciprocal model, vs $R^2 = 0.76$ for the log-log model
        \item There is also a conceptual reason to prefer the reciprocal model: MPG is already a ratio, which tips us off that a reciprocal transformation would be appropriate
        \item The reciprocal model is also simpler to interpret since only one of the variables is transformed
        \item In general, log transformations tend to work best when the spacing between values increases as the values increase (e.g., salaries, city population sizes)
      \end{itemize}
    \end{frame}

    \begin{frame}{Thinking about transformations}
      \begin{itemize}[<+->]
        \item Thinking about whether you want to stretch or squeeze one of the axes, and apply a transformation accordingly (e.g., $\sqrt x$ or $\log x$ to squeeze; $x^2$ or $e^x$ to stretch).
        \item Transformations of $Y$ can address both heteroscedasticity and nonlinearity; transformations of $X$ can only address nonlinearity.
        \item You might need to transform both $X$ and $Y$; if so, start by transforming $Y$ to address the heteroscedasticity, and then transform $X$ to address nonlinearity if necessary.
        \item It's OK to do a little trial and error!
      \end{itemize}
    \end{frame}
  \end{darkframes}

\end{document}
