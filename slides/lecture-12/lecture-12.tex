\documentclass{beamer}\usepackage[]{graphicx}\usepackage[]{color}
% maxwidth is the original width if it is less than linewidth
% otherwise use linewidth (to make sure the graphics do not exceed the margin)
\makeatletter
\def\maxwidth{ %
  \ifdim\Gin@nat@width>\linewidth
    \linewidth
  \else
    \Gin@nat@width
  \fi
}
\makeatother

\definecolor{fgcolor}{rgb}{1, 0.894, 0.769}
\newcommand{\hlnum}[1]{\textcolor[rgb]{0.824,0.412,0.118}{#1}}%
\newcommand{\hlstr}[1]{\textcolor[rgb]{1,0.894,0.71}{#1}}%
\newcommand{\hlcom}[1]{\textcolor[rgb]{0.824,0.706,0.549}{#1}}%
\newcommand{\hlopt}[1]{\textcolor[rgb]{1,0.894,0.769}{#1}}%
\newcommand{\hlstd}[1]{\textcolor[rgb]{1,0.894,0.769}{#1}}%
\newcommand{\hlkwa}[1]{\textcolor[rgb]{0.941,0.902,0.549}{#1}}%
\newcommand{\hlkwb}[1]{\textcolor[rgb]{0.804,0.776,0.451}{#1}}%
\newcommand{\hlkwc}[1]{\textcolor[rgb]{0.78,0.941,0.545}{#1}}%
\newcommand{\hlkwd}[1]{\textcolor[rgb]{1,0.78,0.769}{#1}}%
\let\hlipl\hlkwb

\usepackage{framed}
\makeatletter
\newenvironment{kframe}{%
 \def\at@end@of@kframe{}%
 \ifinner\ifhmode%
  \def\at@end@of@kframe{\end{minipage}}%
  \begin{minipage}{\columnwidth}%
 \fi\fi%
 \def\FrameCommand##1{\hskip\@totalleftmargin \hskip-\fboxsep
 \colorbox{shadecolor}{##1}\hskip-\fboxsep
     % There is no \\@totalrightmargin, so:
     \hskip-\linewidth \hskip-\@totalleftmargin \hskip\columnwidth}%
 \MakeFramed {\advance\hsize-\width
   \@totalleftmargin\z@ \linewidth\hsize
   \@setminipage}}%
 {\par\unskip\endMakeFramed%
 \at@end@of@kframe}
\makeatother

\definecolor{shadecolor}{rgb}{.97, .97, .97}
\definecolor{messagecolor}{rgb}{0, 0, 0}
\definecolor{warningcolor}{rgb}{1, 0, 1}
\definecolor{errorcolor}{rgb}{1, 0, 0}
\newenvironment{knitrout}{}{} % an empty environment to be redefined in TeX

\usepackage{alltt}
\usepackage{../371g-slides}
\title{Inference for Simple Regression 2}
\subtitle{Lecture 12}
\author{STA 371G}
\IfFileExists{upquote.sty}{\usepackage{upquote}}{}
\begin{document}
  
  
  

  \frame{\maketitle}

  % Show outline at beginning of each section
  \AtBeginSection[]{
    \begin{frame}<beamer>
      \tableofcontents[currentsection]
    \end{frame}
  }

  %%%%%%% Slides start here %%%%%%%

  \begin{darkframes}
    \begin{frame}
      \begin{center}
        TX Votes presentation
      \end{center}
    \end{frame}
    
    \begin{frame}{A few notes about the project}
      \begin{itemize}
        \item You'll build up your paper over time---for the Preliminary Proposal all you need are the title, authors, introduction, and data description
        \item Also submit your R script and codebook
        \item Your codebook should include all of the information from the original data set's codebook for each of the variables you include
        \item Data sets and codebooks are posted in Canvas under Files
        \item Think about what you want to predict/model, and then come up with an (inital) list of 8-20 explanatory variables that might be helpful in predicting it
      \end{itemize}
    \end{frame}

    \begin{frame}
      In finance, the $\beta$ of an asset indicates its volatility relative to the market. An asset with:
      \begin{itemize}
        \item $\beta=1$ rises and falls with the market as a whole.
        \item $\beta>1$ is \textbf{more} volatile than the market as a whole.
        \item $\beta<1$ is \textbf{less} volatile than the market as a whole.
      \end{itemize}
      $\beta$ is just the slope of the regression line (i.e. $\hat\beta_1$) when we regress the asset's weekly returns against the weekly returns of a market index.
    \end{frame}

    \begin{frame}[fragile]
\begin{knitrout}
\definecolor{shadecolor}{rgb}{0.137, 0.137, 0.137}\color{fgcolor}
\includegraphics[width=\maxwidth]{/tmp/figures/unnamed-chunk-3-1} 

\end{knitrout}

      The regression line is
      \[
        \widehat{\text{PG}} = 0.11 + 0.45 \cdot\text{W5000},
      \]
      with $R^2=0.15$ and $p=\ensuremath{4.81\times 10^{-11}}$.
      
    \end{frame}

    \begin{frame}{Interpreting the regression statistics}
      \begin{itemize}[<+->]
        \item $\hat\beta_1=0.45$ (``$\beta$'') is the predicted increase in returns for PG when W5000 returns increase by 1 percentage point---since this is $<1$ PG will swing less than the market as a whole
        \item $R^2=0.15$ indicates how closely PG tracks W5000 (the market as a whole)
        \item $p=\ensuremath{4.81\times 10^{-11}}$ tells us whether we can reject the null hypothesis that PG does not move with the market at all \pause (we can! since $p$ is small)
      \end{itemize}
    \end{frame}

    \begin{frame}{Simple regression assumptions for inference}
      We need four things to be true for statistical inference (i.e., hypothesis tests, $p$-values, confidence intervals) to work for regression:
      \pause
      \begin{enumerate}
        \item The errors are independent.
        \item $Y$ is a linear function of $X$ (except for the errors).
        \item The errors are normally distributed.
        \item The variance of $Y$ is the same for any value of $X$ (``homoscedasticity'').
      \end{enumerate}
    \end{frame}

    \begin{frame}{Assumption 1: Independence of errors}
      Independence means that knowing the error (over-/under-prediction by the regression line) for one case doesn't tell you anything about the error for another case.
    \end{frame}

    \begin{frame}{Assumption 1: Independence of errors}
      \begin{itemize}[<+->]
        \item From last time: Knowing how much more Bob drinks than expected (based on the age he started drinking) doesn't give us any suggestion as to how much more Lisa drinks than expected.
        \item From today: Knowing how much better or worse PG performs relative to the market \emph{last} week doesn't tell us anything about how much better or worse PG performs relative to the market \emph{this} week, if we believe the efficient market hypothesis.
        \item \alert{But:} Time-series data often violates the independence assumption!
        \item We can often only verify this assumption by thinking about the situation conceptually.
      \end{itemize}
    \end{frame}

    \begin{frame}{Simple regression assumptions for inference}
      We need four things to be true for statistical inference (i.e., hypothesis tests, $p$-values, confidence intervals) to work for regression:
      \begin{enumerate}
        \item The errors are independent. \greencheckmark
        \item $Y$ is a linear function of $X$ (except for the errors).
        \item The variance of $Y$ is the same for any value of $X$ (``homoscedasticity'').
        \item The errors are normally distributed.
      \end{enumerate}
    \end{frame}


    \begin{frame}{Simple regression assumptions for inference}
      We need four things to be true for statistical inference (i.e., hypothesis tests, $p$-values, confidence intervals) to work for regression:
      \begin{enumerate}
        \item The errors are independent. \greencheckmark
        \item $Y$ is a linear function of $X$ (except for the errors). \greencheckmark
        \item The variance of $Y$ is the same for any value of $X$ (``homoscedasticity''). \greencheckmark
        \item The errors are normally distributed.
      \end{enumerate}
    \end{frame}

    \begin{frame}[fragile]{Assumption 4: Errors are normally distributed}
      Step 1: Look at a histogram of the residuals and ensure they are approximately normally distributed:
\begin{knitrout}
\definecolor{shadecolor}{rgb}{0.137, 0.137, 0.137}\color{fgcolor}\begin{kframe}
\begin{alltt}
\hlkwd{hist}\hlstd{(}\hlkwd{residuals}\hlstd{(model),} \hlkwc{col}\hlstd{=}\hlstr{'darkred'}\hlstd{,} \hlkwc{xlab}\hlstd{=}\hlstr{'Residuals'}\hlstd{)}
\end{alltt}
\end{kframe}
\includegraphics[width=\maxwidth]{/tmp/figures/unnamed-chunk-4-1} 

\end{knitrout}
    \end{frame}

    \begin{frame}[fragile]{Assumption 4: Errors are normally distributed}
      Step 2: Look at a Q-Q plot of the residuals and look for an approximately straight line:
\begin{knitrout}
\definecolor{shadecolor}{rgb}{0.137, 0.137, 0.137}\color{fgcolor}\begin{kframe}
\begin{alltt}
\hlkwd{qqnorm}\hlstd{(}\hlkwd{residuals}\hlstd{(model))}
\end{alltt}
\end{kframe}
\includegraphics[width=\maxwidth]{/tmp/figures/unnamed-chunk-5-1} 

\end{knitrout}
    \end{frame}

    \begin{frame}{Simple regression assumptions for inference}
      We need four things to be true for statistical inference (i.e., hypothesis tests, $p$-values, confidence intervals) to work for regression:
      \begin{enumerate}
        \item The errors are independent. \greencheckmark
        \item $Y$ is a linear function of $X$ (except for the errors). \greencheckmark
        \item The variance of $Y$ is the same for any value of $X$ (``homoscedasticity''). \greencheckmark
        \item The errors are normally distributed. \greencheckmark
      \end{enumerate}
      \pause
      \alert{We always need to check these assumptions before interpreting $p$-values or confidence intervals!}
    \end{frame}
  \end{darkframes}

\end{document}
